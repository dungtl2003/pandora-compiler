\paragraph{Câu lệnh khai báo (\textit{Declaration statement})}

\regexdeclstmt

Câu lệnh khai báo là câu lệnh dùng để tạo ra một tên mới trong chương trình. 

\noindent\textbf{Câu lệnh khai báo biến (\textit{Variable declaration statement})}

\regexvardeclstmt

Câu lệnh khai báo biến dùng để tạo ra một biến mới trong chương trình. Biến là một vùng nhớ được dùng để lưu trữ dữ liệu. Mỗi biến sẽ có một kiểu dữ liệu và một tên riêng. Câu lệnh khai báo biến bắt đầu bằng từ khóa \kw{var}, sau đó là tên biến và kiểu dữ liệu của biến. Ta có thể gán giá trị cho biến ngay sau khi khai báo hoặc sau đó. Nếu ta muốn biến đó thay đổi giá trị, ta có thể khai báo biến với từ khóa \kw{mut} ở trước tên biến. Câu lệnh khai báo biến kết thúc bằng dấu chấm phẩy '\textbf{;}'.

\noindent\textbf{Câu lệnh khai báo hàm (\textit{Function declaration statement})}

\noindent\textbf{Câu lệnh khai báo lớp (\textit{Class declaration statement})}

\noindent\textbf{Câu lệnh khai báo giao diện (\textit{Interface declaration statement})}
