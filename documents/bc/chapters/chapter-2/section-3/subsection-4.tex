\subsection{Câu lệnh (\textit{Statement})}

Câu lệnh là một đơn vị mã thể hiện một hành động sẽ được thực hiện. Một chương trình được tạo thành từ một chuỗi các câu lệnh. Câu lệnh có thể là một lệnh khai báo, lệnh điều kiện, lệnh lặp, ... Các câu lệnh có thể chứa các thành phần bên trong, chẳng hạn như các biểu thức. Biểu thức tạo ra các giá trị, sau đó có thể được sử dụng trong các câu lệnh (ta sẽ nói thêm về biểu thức ở phần \textbf{\ref{ch2:expr}}). Ngoài ra, các câu lệnh có thể chứa các câu lệnh con, tạo thành một câu lệnh lồng nhau. Trong ngôn ngữ lập trình Pandora, ta sẽ có các loại câu lệnh sau:

\regexstmt

\input{chapters/chapter-2/section-3/stmt/decl_stmt.tex}
\input{chapters/chapter-2/section-3/stmt/expr_stmt.tex}
\input{chapters/chapter-2/section-3/stmt/block_stmt.tex}
\input{chapters/chapter-2/section-3/stmt/ctrl_stmt.tex}
\input{chapters/chapter-2/section-3/stmt/import_stmt.tex}
\input{chapters/chapter-2/section-3/stmt/empty_stmt.tex}
