\subsection{Từ tố (\textit{token})}
    Những từ tố đơn giản như dấu toán học sẽ được định nghĩa bằng cách đưa trực tiếp mẫu của nó. Những từ tố phức tạp hơn sẽ được định nghĩa bằng biểu thức chính quy như tên, xâu, số và chú thích. Phần phân tích từ vựng (lexical analyzer) phải có khả năng phân tích chương trình nguồn và đưa ra được các từ tố như sau:

\begin{longtable}{|p{4cm}|p{4cm}|}
    \caption{Bảng tên định danh} \\
\hline
\textbf{\textit{Từ tố}} & \textbf{\textit{Ký hiệu}} \\
\hline
Ident & identifier \\
\hline
RawIdent & raw identifier \\
\hline
\end{longtable}

\begin{longtable}{|p{4cm}|p{4cm}|}
    \caption{Bảng các ký hiệu đặc biệt} \\
\hline
Eq & \kw{=} \\
\hline
Lt & \kw{<} \\
\hline
Le & \kw{<=} \\
\hline
EqEq & \kw{==} \\
\hline
Ne & \kw{!=} \\
\hline
Ge & \kw{>=} \\
\hline
Gt & \kw{>} \\
\hline
AndAnd & \kw{\&\&} \\
\hline
OrOr & \kw{$|\,|$} \\
\hline
Not & \kw{!} \\
\hline
Tilde & \kw{-} \\
\hline
Plus & \kw{+} \\
\hline
Minus & \kw{-} \\
\hline
Star & \kw{*} \\
\hline
Slash & \kw{/} \\
\hline
Precent & \kw{\%} \\
\hline
Caret & \kw{$\wedge$} \\
\hline
And & \kw{\&} \\
\hline
Or & \kw{|} \\
\hline
Shl & \kw{<}\kw{<} \\
\hline
Shr & \kw{>}\kw{>} \\
\hline
PlusEq & \kw{+=} \\
\hline
MinusEq & \kw{-=} \\
\hline
StarEq & \kw{*=} \\
\hline
SlashEq & \kw{/=} \\
\hline
PercentEq & \kw{\%=} \\
\hline
CaretEq & \kw{$\wedge$=} \\
\hline
AndEq & \kw{\&=} \\
\hline
OrEq & \kw{|=} \\
\hline
ShlEq & \kw{<}\kw{<=} \\
\hline
ShrEq & \kw{>}\kw{>=} \\
\hline
Dot & \kw{.} \\
\hline
Comma & \kw{,} \\
\hline
Semicolon & \kw{;} \\
\hline
Colon & \kw{:} \\
\hline
PathSep & \kw{::} \\
\hline
RArrow & \kw{->} \\
\hline
Question & \kw{?} \\
\hline
OpenParen & \kw{(} \\
\hline
CloseParen & \kw{)} \\
\hline
OpenBracket & \kw{[} \\
\hline
CloseBracket & \kw{]} \\
\hline
OpenBrace & \kw{\{} \\
\hline
CloseBrace & \kw{\}} \\
\hline
\end{longtable}

\begin{longtable}{|p{4cm}|p{4cm}|}
    \caption{Bảng các ký hiệu khác} \\
\hline
\textbf{\textit{Từ tố}} & \textbf{\textit{Ký hiệu}} \\
\hline
Bool & boolean \\
\hline
Char & character \\
\hline
Int & integer \\
\hline
Float & float \\
\hline
Str & string \\
\hline
RawStr & raw string \\
\hline
Comment & comment \\
\hline
\end{longtable}

Các từ khóa trong Pandora đều được tính là loại từ tố định danh (\textbf{Ident}), và với mỗi từ khóa, chúng sẽ được gán một ký hiệu đặc biệt tương ứng. Dưới đây là bảng các từ khóa và ký hiệu của chúng:

\begin{longtable}{|p{4cm}|p{4cm}|}
    \caption{Bảng các từ khóa} \\
\hline
\textbf{\textit{Từ khóa}} & \textbf{\textit{Ký hiệu}} \\
\hline
True & \kw{true} \\
\hline
False & \kw{false} \\
\hline
Set & \kw{set} \\
\hline
Mut & \kw{mut} \\
\hline
When & \kw{when} \\
\hline
Alt & \kw{alt} \\
\hline
Fun & \kw{fun} \\
\hline
Br & \kw{br} \\
\hline
Skip & \kw{skip} \\
\hline
For & \kw{for} \\
\hline
In & \kw{in} \\
\hline
During & \kw{during} \\
\hline
As & \kw{as} \\
\hline
Add & \kw{add} \\
\hline
Yeet & \kw{yeet} \\
\hline
\end{longtable}


    Từ vựng của Pandora được xác định không chỉ từ chữ cái (alphabet) và chữ số (digit), mà là hầu hết các ký tự trong bảng mã Unicode \cite{allen2012unicode}, cụ thể như sau:

    \regexdigit

    \regexalphabet

\noindent và các ký tự khác trong bảng mã Unicode.

    Khoảng trắng (\textbf{whitespace}) là bất kỳ chuỗi không trống nào chỉ chứa các ký tự có thuộc tính Unicode Pattern\_White\_Space \cite{web:unicode:report}, cụ thể là:
    \begin{itemize}
        \item{U+0009 (horizontal tab, \kw{\textbackslash t})}
        \item{U+000A (line feed, \kw{\textbackslash n})}
        \item{U+000B (vertical tab)}
        \item{U+000C (form feed)}
        \item{U+000D (carriage return, \kw{\textbackslash r})}
        \item{U+0020 (space, \kw{' '})}
        \item{U+0085 (next line)}
        \item{U+200E (left-to-right mark)}
        \item{U+200F (right-to-left mark)}
        \item{U+2028 (line separator)}
        \item{U+2029 (paragraph separator)}
    \end{itemize}
\noindent Pandora là một ngôn ngữ "dạng tự do", có nghĩa là tất cả các dạng khoảng trắng chỉ dùng để phân tách các từ tố trong ngữ pháp và không có ý nghĩa ngữ nghĩa.

    Tên \textbf{\textit{identifier}} trong Pandora được phân làm hai loại: \textbf{non-keyword identifier} hoặc \textbf{raw identifier}. \textbf{non-keyword identifier} được tạo thành từ tập các kí tự nêu trên và không được là từ khóa (\textbf{keyword}). Trong khi đó, \textbf{raw identifier} có thể là tên hoặc từ khóa (có \kw{r\#} ở phía trước để phân biệt với tên thường hay từ khóa), cụ thể như sau:

    \regexidentifier

\noindent Trong đó \textbf{XID\_Start} và \textbf{XID\_Continue} là các thuộc tính của ký tự trong Unicode liên quan đến tên (định danh). Chúng thường được sử dụng để xác định liệu một ký tự có thể là phần đầu hoặc phần thân của một định danh trong các ngôn ngữ lập trình hay không. Ngoài ra, một tên còn có thể chứa các biểu tượng cảm xúc (\textbf{EMOJI\_SYMBOL}).

\noindent Ví dụ về các identifier và raw identifier:
\begin{lstlisting}[]
abc, _true, a1, a_b // identifier
r#abc, r#true, r#a1, r#a_b // raw identifier
\end{lstlisting}

    Một ký tự chữ (\textbf{character}) là một ký tự Unicode đơn được đặt trong hai ký tự \kw{'} (dấu nháy đơn - U+0027), ngoại trừ chính U+0027, ký tự này phải được thoát bằng ký tự U+005C trước đó (\kw{\textbackslash}), cụ thể như sau:

    \regexcharliteral

\noindent Ví dụ về các ký tự chữ:
\begin{lstlisting}[]
'a', 'b', 'c', '1', '2', '3', '!', '\'', '\"', '(', ')', '_', '+', '-', '=', '{', '}', '[', ']', '|', '\\', ':', ';', '"', '<', '>', ',', '.', '?', '/', ' ', '\t', '\n', '\r'
\end{lstlisting}

    Chuỗi ký tự (\textbf{string literal}) là một chuỗi gồm bất kỳ ký tự Unicode nào được đặt trong hai ký tự U+0022 (\kw{"} - dấu ngoặc kép), ngoại trừ chính U+0022, ký tự này phải được thoát bằng ký tự U+005C trước đó (\kw{\textbackslash}). Chuỗi ký tự cho phép có các line-breaks (cho phép ngắt dòng, được biểu thị bởi ký tự U+000A). Khi ký tự U+005C không thoát (\kw{\textbackslash}) xuất hiện ngay trước ngắt dòng, ngắt dòng sẽ không xuất hiện trong xâu được biểu diễn trong từ tố, cụ thể như sau:

    \regexstringliteral

\noindent Ví dụ về các chuỗi ký tự:
\begin{lstlisting}[]
"abc", "a b c", "a\nb\nc", "a\"b\"c", "a\\b\\c", "a\tb\tc", "a\rb\rc"
// special case (this will be like "ab")
"a\

    b"
\end{lstlisting}

    Ta có thể thấy, các ký tự chữ hoặc các xâu có thể chứa một hoặc một vài loại \textbf{escape} (thoát ký tự). Một escape bắt đầu bằng U+005C (\kw{\textbackslash}) và tiếp tục bằng một trong các dạng sau:

    \begin{itemize}
    \item{\textbf{Whitespace escape} (thoát khoảng trắng) là một trong các ký tự U+006E (\kw{n}), U+0072 (\kw{r}) hoặc U+0074 (\kw{t}), biểu thị các giá trị Unicode U+000A (\kw{LF}), U+000D (\kw{CR}) hoặc U+0009 (\kw{HT}) tương ứng}
    \item{\textbf{Null escape} (thoát null) là ký tự U+0030 (\kw{0}) và biểu thị giá trị Unicode U+0000 (\kw{NUL})}
    \item{\textbf{Backslash escape} (thoát gạch chéo ngược) là ký tự U+005C (\kw{\textbackslash}) phải được thoát để biểu thị chính nó}
    \end{itemize}

    Chuỗi ký tự thô (\textbf{raw string literal}) không xử lý bất kỳ escape nào. Chúng bắt đầu bằng ký tự U+0072 (\kw{r}), theo sau là ít hơn 256 ký tự U+0023 (\kw{\#}) và ký tự U+0022 (\kw{"} - dấu nháy kép). Raw string có thể chứa chuỗi bất kì các ký tự nào trong Unicode. Nó chỉ được kết thúc bởi một ký tự U+0022 (\kw{"} - dấu nháy kép) khác, theo sau là các ký tự U+0023 (\kw{\#}) có số lượng giống với các ký tự U+0023 (\kw{\#}) đứng trước ký tự U+0022 (\kw{"} - dấu nháy kép) mở đầu. Tất cả các ký tự Unicode đều có trong phần thân raw string thể hiện chính chúng, các ký tự U+0022 (\kw{"} - dấu nháy kép) (trừ khi được theo sau bởi ít nhất nhiều ký tự U+0023 (\kw{\#}) như đã được sử dụng để bắt đầu raw string) hoặc U+005C (\kw{\textbackslash}) không có ý nghĩa gì đặc biệt. Cụ thể như sau:

    \regexrawstringliteral

\noindent Trong đó, cần lưu ý rằng \textbf{non-greedy behavior} (hành vi không tham lam) được sử dụng để xác định phần thân raw string. Điều này có nghĩa là phần thân raw string sẽ kết thúc ngay khi có thể, không cần phải chờ đến khi có thể kết thúc chuỗi raw string.

\noindent Ví dụ về các chuỗi ký tự thô:
\begin{lstlisting}[]
r"abc", r"a\b\c", r#"a"b"c"#, r#"a###"b"c"#
\end{lstlisting}

    Một hằng số nguyên (\textbf{integer literal}) có một trong bốn dạng sau:

    \begin{itemize}
        \item{Một hằng \textbf{thập phân} bắt đầu bằng một chữ số thập phân và tiếp tục bằng bất kỳ hỗn hợp nào của các chữ số thập phân và dấu gạch dưới U+005F (\kw{\_})}
        \item{Một hằng \textbf{thập lục phân} bắt đầu bằng chuỗi ký tự U+0030 U+0068 (\kw{0h}) hoặc chuỗi ký tự U+0030 U+0048 (\kw{0H}) và tiếp tục dưới dạng bất kỳ hỗn hợp nào (có ít nhất một chữ số) gồm các chữ số thập lục phân và dấu gạch dưới U+005F (\kw{\_})}
        \item{Một chữ bát phân bắt đầu bằng chuỗi ký tự U+0030 U+006F (\kw{0o}) hoặc chuỗi ký tự U+0030 U+004F (\kw{0O}) và tiếp tục dưới dạng bất kỳ hỗn hợp nào (có ít nhất một chữ số) gồm các chữ số bát phân và dấu gạch dưới U+005F (\kw{\_})}
        \item{Một chữ số nhị phân bắt đầu bằng chuỗi ký tự U+0030 U+0062 (\kw{0b}) hoặc chuỗi ký tự U+0030 U+0042 (\kw{0B})và tiếp tục dưới dạng bất kỳ hỗn hợp nào (có ít nhất một chữ số) gồm các chữ số nhị phân và dấu gạch dưới U+005F (\kw{\_})}
    \end{itemize}

    \regexintegerliteral

\noindent Ví dụ về các hằng số nguyên:
\begin{lstlisting}[]
0, 3, 124, 1_000_000, 1__ // decimal
0h0, 0h3, 0H1_24, 0H1_000_000, 0h___1__ // hexadecimal
0o0, 0o3, 0O1_24, 0O1_000_000, 0o___1__ // octal
0b0, 0b1, 0B1_24, 0B1_000_000, 0b___1__ // binary
\end{lstlisting}

    Một hằng số thực (\textbf{float literal}) có một trong hai dạng:
    \begin{itemize}
        \item{Một số thập phân theo sau là ký tự dấu chấm U+002E (\kw{.}). Theo sau có thể là một số thập phân khác (sau số thập phân đó có thể có số mũ)}
        \item{Một số thập phân theo sau là số mũ}
    \end{itemize}

    \regexfloatliteral
    
\noindent Ví dụ về các hằng số thực:
\begin{lstlisting}[]
0.0, 3.14, 124.25_3, 1_000_000.0, 1__.0, 0.0e0, 3.0e3__, 
124.3_e__2__, 1_000_000.0+e32, 1__.0_-E-2
\end{lstlisting}

    Chú thích không phải tài liệu (\textbf{non-doc comment}) có thể là dòng (\kw{//}) hoặc là khối (\kw{/* ... */}). Pandora có hỗ trợ các chú thích khối lồng nhau. Các chú thích kiểu này được hiểu là một dạng khoảng trắng.

    Chú thích tài liệu (\textbf{doc comment}) được chia làm hai loại chính: chú thích tài liệu ngoài (\textbf{outer doc comment}) và chú thích tài liệu trong (\textbf{inner doc comment}). Chú thích tài liệu dòng bên ngoài sẽ bắt đầu bằng chuỗi ký tự U+002F U+002F U+0040 (\kw{//@}), còn khối bên ngoài sẽ có dạng U+002F U+002A U+0040 ... U+002A U+002F (\kw{/*@ ... */}). Trong khi đó, chú thích tài liệu dòng bên trong là U+002F U+002F U+0021 (\kw{//!}), khối bên trong có dạng U+002F U+002A U+0021 ... U+002A U+002F (\kw{/*! ... */}). Việc có thêm loại chú thích này giúp Pandora hỗ trợ việc tạo tài liệu API tự động một cách dễ dàng.

    Các chú thích được biểu diễn như sau:

    \regexlinecomment

    \regexblockcomment

    \regexdoc

\noindent Ví dụ về các chú thích:
\begin{lstlisting}[]
// this is a line comment

/* this is 
a block comment
/* this is a nested
block comment */ */

//@ this is an outer doc comment

/*! this is an 
inner doc comment 
/*@ this is a nested 
outer doc comment */ */
\end{lstlisting}

