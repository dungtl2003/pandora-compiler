\subsubsection{Câu lệnh khối (\textit{Block statement})}

Câu lệnh khối là một nhóm các câu lệnh được đặt trong một cặp dấu ngoặc nhọn U+007B U+007D (\textbf{\{ \}}). Câu lệnh khối được sử dụng để nhóm các câu lệnh lại với nhau, tạo thành một khối mã thực thi. Câu lệnh khối có thể chứa một hoặc nhiều câu lệnh bên trong, cũng có thể không chứa câu lệnh nào. Câu lệnh khối thường được sử dụng trong các cấu trúc điều kiện hoặc vòng lặp. Đôi khi, chúng cũng được sử dụng để giới hạn phạm vi của biến. Câu lệnh khối được mô tả bởi biểu thức chính quy sau:

\label{ch2:block_stmt}
\regexblockstmt

\noindent Ví dụ về câu lệnh khối:
\begin{lstlisting}[]
{
    set a: int = 5;
    set b: int = 3;

    {
        set c: int = a + b;
        println(c as str);
    }
}
\end{lstlisting}
