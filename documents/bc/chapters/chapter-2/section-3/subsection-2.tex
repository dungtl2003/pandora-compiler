\subsection{Câu lệnh (\textit{Statement})}

Câu lệnh là một đơn vị mã thể hiện một hành động sẽ được thực hiện. Một chương trình được tạo thành từ một chuỗi các câu lệnh. Câu lệnh có thể là một lệnh khai báo, lệnh điều kiện, lệnh lặp, ... Các câu lệnh có thể chứa các thành phần bên trong, chẳng hạn như các biểu thức. Biểu thức tạo ra các giá trị, sau đó có thể được sử dụng trong các câu lệnh (ta sẽ nói thêm về biểu thức ở phần \textbf{\ref{ch2:expr}}). Ngoài ra, các câu lệnh có thể chứa các câu lệnh con, tạo thành một câu lệnh lồng nhau. Trong ngôn ngữ lập trình Pandora, ta sẽ có các loại câu lệnh sau:

\regexstmt

\subsubsection{Câu lệnh khai báo (\textit{Declaration statement})}

Câu lệnh khai báo là câu lệnh dùng để tạo ra một tên mới trong chương trình. Ta sẽ có hai loại khai báo sau: khai báo biến và khai báo hàm.

\regexdeclstmt

\noindent\textbf{\label{ch2:decl_var_stmt}Câu lệnh khai báo biến (\textit{Variable declaration statement})}

    Câu lệnh khai báo biến dùng để tạo ra một biến mới trong chương trình. Biến là một vùng nhớ được dùng để lưu trữ dữ liệu. Mỗi biến sẽ có một kiểu dữ liệu và một tên riêng. Câu lệnh khai báo biến bắt đầu bằng từ khóa \kw{set}, sau đó là tên biến và kiểu dữ liệu của biến. Ta có thể gán giá trị cho biến ngay sau khi khai báo hoặc sau đó. Nếu ta muốn biến đó thay đổi giá trị, ta có thể khai báo biến với từ khóa \kw{mut} ở trước tên biến. Câu lệnh khai báo biến kết thúc bằng dấu chấm phẩy U+003B (\kw{;}). Ta có thể tạo một biến mới với tên trùng với tên của biến đã tồn tại trong cùng một phạm vi (\textbf{shadow variable}). Khi này, biến mới sẽ ghi đè lên biến cũ. Câu lệnh khai báo biến được mô tả bởi biểu thức chính quy sau:

\regexvardeclstmt

\noindent Ví dụ về câu lệnh khai báo biến:
\begin{lstlisting}[]
set a: int = 5;
set b: int;
set mut c: str = "hello world";
set mut a: float = 3.14;
\end{lstlisting}

\noindent\textbf{\label{ch2:decl_func_stmt}Câu lệnh khai báo hàm (\textit{Function declaration statement})}

    Câu lệnh khai báo hàm dùng để tạo ra một hàm mới trong chương trình. Hàm là một khối mã thực thi một tập hợp các công việc cụ thể. Câu lệnh khai báo hàm bắt đầu bằng từ khóa \kw{fun}, sau đó là tên hàm, danh sách tham số, kiểu trả về và thân hàm. Nếu hàm không trả về giá trị, kiểu trả về sẽ là \kw{unit}. Câu lệnh khai báo hàm kết thúc bằng dấu chấm phẩy U+003B (\kw{;}). Câu lệnh khai báo hàm được mô tả bởi biểu thức chính quy sau:

\regexfuncdeclstmt

\noindent Ví dụ về câu lệnh khai báo hàm:
\begin{lstlisting}[]
fun add(a: int, mut b: int) -> int {
    b += 2;
    yeet a + b;
}

fun hello(name: str) {
    println("Hello, " + name);
}
\end{lstlisting}

\subsubsection{Câu lệnh biểu thức (\textit{Expression statement})}            
\label{ch2:expr_stmt}
    Câu lệnh biểu thức là câu lệnh dùng để thực thi một biểu thức nào đó. Biểu thức có thể là một biểu thức gán giá trị, một biểu thức gọi hàm hoặc một biểu thức toán tử. Câu lệnh biểu thức kết thúc bằng dấu chấm phẩy U+003B (\textbf{;}). Câu lệnh biểu thức được mô tả bởi biểu thức chính quy sau:

\regexexprstmt

\noindent Ví dụ về câu lệnh biểu thức:
\begin{lstlisting}[]
5;
4 + 3;
some_func();
\end{lstlisting}

\subsubsection{Câu lệnh khối (\textit{Block statement})}

Câu lệnh khối là một nhóm các câu lệnh được đặt trong một cặp dấu ngoặc nhọn U+007B U+007D (\textbf{\{ \}}). Câu lệnh khối được sử dụng để nhóm các câu lệnh lại với nhau, tạo thành một khối mã thực thi. Câu lệnh khối có thể chứa một hoặc nhiều câu lệnh bên trong, cũng có thể không chứa câu lệnh nào. Câu lệnh khối thường được sử dụng trong các cấu trúc điều kiện hoặc vòng lặp. Đôi khi, chúng cũng được sử dụng để giới hạn phạm vi của biến. Câu lệnh khối được mô tả bởi biểu thức chính quy sau:

\label{ch2:block_stmt}
\regexblockstmt

\noindent Ví dụ về câu lệnh khối:
\begin{lstlisting}[]
{
    set a: int = 5;
    set b: int = 3;

    {
        set c: int = a + b;
        println(c as str);
    }
}
\end{lstlisting}

\subsubsection{Câu lệnh điều khiển (\textit{Control flow statement})}

    Câu lệnh điều khiển là câu lệnh dùng để kiểm soát luồng thực thi của chương trình. Câu lệnh điều khiển bao gồm các câu lệnh rẽ nhánh và lặp. Câu lệnh điều khiển được sử dụng để thực hiện các công việc như kiểm tra điều kiện, lặp lại một nhóm câu lệnh nào đó, thoát khỏi vòng lặp, ... Câu lệnh điều khiển được mô tả bởi biểu thức chính quy sau:

\regexctrlstmt

\noindent\textbf{\label{ch2:if_stmt}Câu lệnh rẽ nhánh (\textit{If statement})}

    Câu lệnh rẽ nhánh là câu lệnh dùng để kiểm tra một biểu thức điều kiện. Nếu biểu thức điều kiện đúng, chương trình sẽ thực thi một nhóm câu lệnh nào đó. Nếu biểu thức điều kiện sai, chương trình sẽ thực thi một nhóm câu lệnh khác. Câu lệnh rẽ nhánh bắt đầu bằng từ khóa \kw{when}, sau đó là biểu thức điều kiện cần kiểm tra. Nếu biểu thức điều kiện đúng, chương trình sẽ thực thi câu lệnh nằm trong dấu ngoặc nhọn U+007B U+007D (\kw{\{ \}}) ngay sau. Nếu biểu thức điều kiện sai, chương trình sẽ thực thi câu lệnh nằm trong dấu ngoặc nhọn ngay sau từ khóa \kw{alt}. Câu lệnh rẽ nhánh có thể không chứa câu lệnh nào ở một trong hai nhánh. Biểu thức điều kiện bắt buộc phải là một biểu thức trả về giá trị đúng sai (boolean). Câu lệnh rẽ nhánh không bắt buộc phải có nhánh \kw{alt}. Câu lệnh rẽ nhánh được mô tả bởi biểu thức chính quy sau:

\regexifstmt

\noindent Ví dụ về câu lệnh rẽ nhánh:
\begin{lstlisting}[]
when a > b {
    println("a is greater than b");
} alt when a < b {
    println("a is less than b");
} alt {
    println("a is equal to b");
}
\end{lstlisting}

\noindent\textbf{Câu lệnh lặp (\textit{Loop statement})}

Câu lệnh lặp là câu lệnh dùng để lặp lại một nhóm câu lệnh nào đó nhiều lần.

\regexloopstmt

\noindent\textbf{\label{ch2:while_stmt}Câu lệnh lặp biểu thức điều kiện (\textit{Predicate loop statement})}

    Câu lệnh lặp biểu thức điều kiện là câu lệnh dùng để lặp lại một nhóm câu lệnh nào đó nhiều lần dựa trên một biểu thức điều kiện. Câu lệnh lặp biểu thức điều kiện bắt đầu bằng từ khóa \kw{during}, sau đó là biểu thức điều kiện cần kiểm tra. Nếu biểu thức điều kiện đúng, chương trình sẽ thực thi câu lệnh nằm trong dấu ngoặc nhọn U+007B U+007D (\kw{\{ \}}) ngay sau. Nếu biểu thức điều kiện sai, chương trình sẽ thoát khỏi vòng lặp. Biểu thức điều kiện bắt buộc phải là một biểu thức trả về giá trị đúng sai (boolean). Câu lệnh lặp biểu thức điều kiện được mô tả bởi biểu thức chính quy sau:

\regexpredloopstmt

\noindent Ví dụ về câu lệnh lặp biểu thức điều kiện:
\begin{lstlisting}[]
set mut a: int = 0;
during a < 10 {
    println(a as str);
    a += 1;
}
\end{lstlisting}

\noindent\textbf{\label{ch2:for_stmt}Câu lệnh lặp trình lặp (\textit{Iterator loop statement})}

    Câu lệnh lặp trình lặp là câu lệnh dùng để lặp qua một tập hợp các phần tử nào đó. Câu lệnh lặp trình lặp bắt đầu bằng từ khóa \kw{for}, sau đó là biến lặp, từ khóa \kw{in} và tập hợp cần lặp qua. Biến lặp sẽ lấy giá trị của từng phần tử trong tập hợp. Câu lệnh lặp trình lặp có thể chứa một hoặc nhiều câu lệnh bên trong. Biểu thức tập hợp bắt buộc phải là một biểu thức trả về một đối tượng có thể lặp qua (iterable). Câu lệnh lặp trình lặp được mô tả bởi biểu thức chính quy sau:

\regexiterloopstmt

\noindent Ví dụ về câu lệnh lặp trình lặp:
\begin{lstlisting}[]
set arr: [int] = [1, 2, 3, 4, 5];
for i in arr {
    println(i as str);
}

set str: str = "hello";
for c in str {
    println(c as str);
}
\end{lstlisting}

\noindent\textbf{\label{ch2:return_stmt}Câu lệnh trả về (\textit{Return statement})}

    Câu lệnh trả về là câu lệnh dùng để trả về một giá trị từ một hàm. Câu lệnh trả về bắt đầu bằng từ khóa \kw{yeet}, sau đó có thể là giá trị cần trả về. Câu lệnh này chỉ có thể xuất hiện trong hàm. Câu lệnh trả về được mô tả bởi biểu thức chính quy sau:

\regexreturnstmt

\noindent Ví dụ về câu lệnh trả về:
\begin{lstlisting}[]
fun add(a: int, b: int) -> int {
    yeet a + b;
}

fun hello(name: str) {
    println("Hello, " + name);
    yeet;
}
\end{lstlisting}

\noindent\textbf{\label{ch2:break_stmt}Câu lệnh thoát khỏi vòng lặp (\textit{Break statement})}

    Câu lệnh thoát khỏi vòng lặp là câu lệnh dùng để thoát khỏi vòng lặp hiện tại. Câu lệnh thoát khỏi vòng lặp bắt đầu bằng từ khóa \kw{br}. Câu lệnh này chỉ có thể xuất hiện trong vòng lặp. Câu lệnh thoát khỏi vòng lặp được mô tả bởi biểu thức chính quy sau:

\regexbreakstmt

\noindent Ví dụ về câu lệnh thoát khỏi vòng lặp:
\begin{lstlisting}[]
set mut a: int = 0;
during a < 10 {
    println(a as str);
    a += 1;
    when a == 5 {
        br;
    }
}
\end{lstlisting}

\noindent\textbf{\label{ch2:continue_stmt}Câu lệnh tiếp tục vòng lặp (\textit{Continue statement})}

    Câu lệnh tiếp tục vòng lặp là câu lệnh dùng để bỏ qua phần còn lại của vòng lặp hiện tại và tiếp tục với vòng lặp tiếp theo. Câu lệnh tiếp tục vòng lặp bắt đầu bằng từ khóa \kw{skip}. Câu lệnh này chỉ có thể xuất hiện trong vòng lặp. Câu lệnh tiếp tục vòng lặp được mô tả bởi biểu thức chính quy sau:

\regexcontinuestmt

\noindent Ví dụ về câu lệnh tiếp tục vòng lặp:
\begin{lstlisting}[]
set mut a: int = 0;
during a < 10 {
    a += 1;
    when a % 2 == 0 {
        skip;
    }
    println(a as str);
}
\end{lstlisting}

\subsubsection{\label{ch2:import_stmt}Câu lệnh thêm thư viện (\textit{Import statement})}

Câu lệnh thêm thư viện là câu lệnh dùng để bổ sung một module vào chương trình. Câu lệnh này bắt đầu bằng từ khóa \kw{add}, sau đó là tên module cần thêm. Nó được mô tả bởi biểu thức chính quy sau:

\regeximportstmt

\noindent Ví dụ về câu lệnh thêm thư viện:
\begin{lstlisting}[]
add std;
{
    add math;
    std.println("Hello, world!");
    math.sqrt(4);
}

std.println("Hello, world!");
// math.sqrt(4); // error, module math is not imported in this scope
\end{lstlisting}

\subsubsection{\label{ch2:empty_stmt}Câu lệnh rỗng (\textit{Empty statement})}

Câu lệnh rỗng là câu lệnh không thực hiện bất kỳ công việc nào. Câu lệnh rỗng được mô tả bởi biểu thức chính quy sau:

\regexemptystmt

\noindent Ví dụ về câu lệnh rỗng:
\begin{lstlisting}[]
set a: int = 5;
; // empty statement
set b: int = 3;
\end{lstlisting}

